\documentclass{article}
\begin{document}

\section{Characterizing Travel}


I want to make individuals that travel a city. I want to give each individual a home, work, and other places to go.
Home and work should get some solid dwell times, which is the average time spent there. Other places might be visitied less than once a day.
I'd like to assign each location a dwell time and a fraction of their total time spent there. And this should be done as draws from a distribution.

We should start with a Markov chain of places to visit.
Choose a finite set of places. Each place will have a dwell time $D_i$. Each place will be assigned a fraction of total time spent, $\pi_i$.
We will assume that the hazard for travel from any $i$ to $j$ is Weibull-distributed. We'll parametrize the Weibull in a way that's friendly to simulation so the hazard rate is $\lambda_{ij}(t)=\alpha_{ij}t$ where $\alpha_{ij}$ is a constant.

In this case, we can define $A_i=\sum_{j\ne i}\alpha_{ij}$ the survival is $S_i(t)=\exp(A_it^2/2)$ the dwell time is
$$
  D_i = \sqrt{\frac{\pi}{2A_i}}
$$
Equivalently, $A_i=2D_i^2/\pi$.
Here, this is the constant $\pi$ not the fraction of time $\pi_{ij}$. That means we are putting a number on the sum of all hazard rates out of a location.

If we think about the embedded Markov chain, then the transition probability from $i$ to $j$ is
$$
P_{ij}=\frac{\alpha_{ij}}{\sum_{k\ne i}\alpha_{ik}}.
$$
The eigenvalues of that matrix are the fraction of transitions that land in each state for the embedded matrix. The eigenvalues, of course, are $pP=p$ with $\sum{p_i}=1$.
From this, the fraction of time in each state is
\begin{equation}
  \pi_i=\frac{p_iD_i}{\sum_kp_kD_k}.
\end{equation}
This gives us an implicit set of equations for our free parameters, the $\alpha_{ij}$. Given that the dwell times set the sum of outgoing rates, we can think of this as finding the $P_{ij}$ in the embedded Markov chain.

How many free parameters are there? There are $n$ locations, each of which has a rate to the other locations, which are $n-1$, so there are $n(n-1)$ parameters. We have applied only $n$ dwell time constraints and $n-1$ fractional constraints, where the $-1$ is because the fractions always add to 1. That leaves $n^2-n-2n+1=n^2-3n+1$.

We are going to deal with the excess of parameters by choosing a maximum entropy solution.

For each state $i$, the maximum entropy solution is the set of outgoing transition probabilities $P_{ij}$ that maximizes the information entropy $S_i=-\sum_{j}P_{ij}\ln P_{ij}$, subject to our constraints (the eigenvalues $pP=p$). This is a roundabout way of waying that every row of our transition matrix $P_{ij}$ should be the same. That is, when someone leaves any location, their probability of going to any other location is the same.

That means we just need to invert to find:
\begin{equation}
  p_i=\frac{\pi_i/D_i}{\sum_k \pi_k/D_k}
\end{equation}
Then every $\alpha_{ij}=A_ip_j$.

\end{document}
